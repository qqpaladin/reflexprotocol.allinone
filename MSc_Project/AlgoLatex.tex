\documentclass[12pt]{article}
\begin{document}

\section*{Algorithm}

Our PREFLEX balancer maintains individual routing tables associated to each domain egress. For each table entry we associate a flowlet ratio $f_{di}$, which defines the fraction of flowlets to a destination $d$ that should be assigned to route $i$. Now consider only LEX enabled traffic destined to prefix $d$ at a PREFLEX balancer.Let $T_{i}$ be the number of bytes sent through route $i$ for the previous time period. Let $L_{i}$ be the number of bytes marked with the loss experienced codepoint and sent through route $i$ in the previous time period. Let $N$ be the number of available routes for the given destination prefix. Let $L = {\sum_{i} L_{i}}$ and $T = {\sum_{i} T_{i}}$.
\\Splitting traffic may follow distinct approaches.  One tendency is to attempt to equalise utilisation. When small adjustments to traffic splits are preferred, there is a desire to
 be conservative. Finally, there may be the desire to balance losses. Call these splits $f(E)_{i}$, $f(C)i_{i}$ and $f(L)_{i}$ where $E$, $C$ and $L$ stand for “equal”, “conservative” and “loss driven”. Use the dash notation for the same quantities in the next time period. Then our final distribution of traffic across all routes is:
\begin{equation}
f'_{i} = \beta_{E}f'(E)_{i} + \beta_{C}f'(C)_{i} + \beta_{L}f'(L)_{i}
\end{equation}

\\where the $\beta_{\cdot}$ are user set parameters in $\left(0,1\right)$ such that $\beta_{E}+\beta_{C}+\beta_{L} = 1$. Now by definition $f'(E)_{i} = 1/N$ and $f'(C)_{i} = T_{i}/T$
\\ $f'(L)_{i}$ is left. We desire to choose a distribution that will  will equalize the loss ratio $\rho_{i} = L'_{i}/T'_{i}$ for all the permitted routes i. The loss rate is an unknown function of $T_{i}$ and the link bandwidth $B_{i}$. However, it is reasonable to assume that the loss rate is increasing with $T_{i}$ and decreasing with $B_{i}$. Hence, what ever the true function is, we could assume that in a small region around the current values of $T_{i}$ and $B_{i}$, it is locally linear $\rho_{i} = k_{i}T_{i}/B_{i}$, where $k_{i}$ is an unknown constant. Substituting gives
\begin{equation}
B_{i}/k_{i} = T_{i}^{2}/L_{i}
\end{equation}
\\So for the next period of time, loss rate are $\rho'_{i} = T'_{i}/L'_{i} = k'_{i}T'{i}/B'{i}$. Hence, to make the loss rate equal for all routes, we should choose $T_{i}$ that verifies $L'_{i]/T'_{i}=C$, where C is an unknown constant. Therefore we should take  $T'_{i}= CB'_{i}/k'_{i}$ and by assuming we are still near the locally linear region we get  $T'_{i}= CB_{i}/k_{i}$ and hence using (1) we get  $T'_{i}= CT_{i}^{2}/L_{i}$. Summing over i will give us: $T' = C \sum_{i} T_{i}^2/L_{i}$. Assuming that the traffic on average stayed the same we get the value of C and and though: 
\begin{equation}
T'_{i} = \frac {TT_{i}^{2}} {L_{i} \sum_{i} (T_{i}^2/L_{i})}
\end{equation}

Hence, the "loss driven" split, that needs to be set to achieve the $T'_{i}$ number of bytes on path i, is:

\begin{equation}
f'(L)_{i} = \frac{T'_{i}}{T} =  \frac {T_{i}^{2}} {L_{i} \sum_{i} (T_{i}^2/L_{i})}
\end{equation}

\end{document}

