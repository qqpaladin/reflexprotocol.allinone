\begin{abstract}
Taking full advantage of path diversity is one of the most desired properties for the future Internet. It is a cost effective solution for networking with the required robustness and increased capacity to respond the unrelentless demand on Internet services. Despite this it has been difficult to move beyond the single path forwarding model of Internet routing. The advent in recent years of adaptive Traffic Engineering protocol like MATE \cite{elwal1} and TEXCP \cite{kan1} was considered as significant improvement, but both have not been deployed since they provide only local optimization, with uncertain consequences for the end-to-end experience of users.

In parallel, ISPs have been observing in recent years more of their network capacity being monopolized by a small part of their customers and  are struggling to define an accounting measure to charge them based on congestion caused. In order to resolve this problem, the concept of Congestion Exposure \cite{} proposed to make end users reveal explicitly the congestion level to the network. In Re-ECN protocol this information was added to the IP header.

PREFLEX \cite{arau1} is an architecture that uses a simple form of {\it congestion exposure} to pool more efficiently path diversity. It also promotes the cooperation of both end hosts and edge network to invest path diversity. In this project we investigate how this architecture could balance congestion instead of load. The performance of this balancing mode is then compared with the Traffic Engineering protocol TEXCP.
\end{abstract}

