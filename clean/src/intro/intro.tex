\chapter{Introduction}
\label{chapter:intro}

\section{Overview} % section 1.1
Over the past two decades, the number of Internet users as well as the services offered to them has tremendously increased. Consequently, network requirements in terms of traffic amount and quality of service have never been tougher. Traffic Engineering is one of the engineering processes that network providers have always relied on to make their networks answer demands while keeping profitable margins. 

Traffic Engineering deals with the optimization of network utilization. It possesses a set network parameters and mechanisms that it uses to drive it to an optimal state. Routing belongs to this list and is one of the most important. Indeed, the simple but efficient routing of Internet had been always viewed as one of its success reasons. Yet, single path forwarding is an in-built property of Internet routing and is viewed today as a severe restriction. Indeed, path diversity is considered as a necessity to ensure robustness and provides a cost effective solution to boost its capacity.

The advent of MPLS technology mitigated some of these restrictions and opened the door for new possibilities to Traffic Engineering. While, the first generation of MPLS-TE solution were all static and didn't allow the network to adapt to traffic changes, recent proposals like MATE and TEXCP succeed in being dynamic and stable at the same time. All of these Traffic Engineering protocols focused on the optimization of network resources. Because these parameters are usually proper to a single domain, it wasn't easy to generalize their principles to other contexts like inter-domain TE and Multihoming. Moreover, network resources state are not necessary a good indicator of the quality of service perceived by end users.

In parallel, ISPs seeking a metric to fairly charge the customers and control the congestion that the heavy users cause, were struggling from the lack of information available at the network layer. Indeed, one of the oldest and key architecture concepts of Internet is the hourglass model. This model conceive the role of network layer to the simple task of packet forwarding. Limitation of equipments capacity at that time motivated this design. By reducing the network task, there was no more need to reveal more control information the network layer. The concept of Congestion Exposure pointed out this lack and defended that new feedback mechanisms should be enabled at the end hosts so they could reveal to the network the level of the congestion they are experiencing. Re-Ecn protocol is an illustration of the principle and provide the network with a full accounting metric of the congestion caused by each end user. The network will be able then to set incentives for the customers so the heavy one limit the congestion cause and a better quality is achieved to the network.

The question is whether the revealed congestion level could be used by the network for Traffic Engineering and path diversity, and what are the characteristics of this mode compared to the current load balancing techniques. The project of PREFLEX  conducted within Network and Services Research Laboratory in EE department of UCL, works on an architecture for balancing congestion that uses a simple form of Congestion Exposure and promotes cooperation by end hosts and The network.

\section{Objectives}

The aim of this project is to study how PREFLREX architecture could proceed to balance the traffic according to the congestion. The balancing algorithm should answer multiple requirement imposed by the architecture. 

Then the performance should be analysed and compared with other traditional approaches. TEXCP one of the most advanced Traffic Engineering protocols has been chosen. The comparison between the two should be conducted using system simulations. PREFLEX architecture was already implemented in the network simulator ns3. A TEXCP module needs to be added to the framework.
The new framework will be used to compare the performance and features of the two approaches.

\section{Document Structure}

The remainder of this report is structured as follows. Chapter 2 describes some of the research fields that are examined within this dissertation. Chapter 3 introduces the different components of the architecture PREFLEX as well as its balancing algorithm. Chapter 4 describes the implementation of the framework  developed to conduct the simulations. Chapter 5 presents the result of the simulations and analysis the performance and features of PREFLEX  compared to TEXCP.