\begin{abstract}
Taking full advantage from path diversity is one of the most wanted properties for the future design of Internet. It is a cost effective solution for enabling network with the required robustness and boosted capacity to answer the tremendously increasing demand on Internet services. Yet, it was quite difficult to overtake the single path forwarding model of Internet routing. The advent in recent years of adaptive Traffic Engineering protocol like MATE \cite{elwal1} and TEXCP \cite{kan1} was considered as a breakthrough. But even if they do very well in optimizing a network capacity, they don't allow to directly target  the optimization of end-to-end experience of end users.

In parallel, ISPs were observing in recent years more of their network capacity being monopolized by a small part of their customers and  were struggling to define an accounting measure to charge them based on the congestion they make. In order to resolve this problem, the concept of Congestion Exposure proposed to make end users reveal explicitly the congestion level to the network. In Re-ECN protocol this information was added to the IP header.

PREFLEX \cite{arau1} is an architecture that uses a simple form of {\it congestion exposure} to pool more efficiently path diversity. It also promotes the cooperation of both end hosts and edge network to invest path diversity. In this project we investigate how this architecture could balance congestion instead of load. The performance of this balancing mode is then compared with the Traffic Engineering protocol TEXCP.
\end{abstract}

