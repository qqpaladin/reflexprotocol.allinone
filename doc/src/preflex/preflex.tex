PREFLEX Path Re-feedback with Loss Exposure

This chapter introduces PREFLEX, a new architecture for pooling resources -more particularly path diversity- that promotes cooperation between end hosts and edge network and attempts to draw benefits from Congestion Exposure (see 2.3). Section 3.1 presents Loss EXposure LEX, a congestion exposure protocol that end hosts uses to reveal experienced losses to the network. While in section 3.2, we are going to see Path RE-Feedback PREF the second components of the architecture, that edge network uses to inform the end host of its path preference. This path preference is based on  the split that the balancing algorithm described in section 3.3 chooses to optimize the global congestion experienced in the different paths.

\section{Loss Exposure}

During the presentation of the Congestion Exposure concept, we've discussed how there is a lack of congestion information available at the network layer and how it affects the network efforts towards controlling the congestion and fairly charging the customers for the congestion that they are causing. This lack also affects the optimization process of networks, by restricting mostly their target to the optimization of resources utilization rather than targeting directly the quality of service customers are getting. Keeping the same line of thinking, Loss EXposure is a simple protocol that allows end hosts to signal path loss back to the network. The information about the loss traditionally confined in the transport layer will be revealed by explicitly marking the IP header of retransmitted  packets.

Establishing the feedback

End hosts need to establish a connection first, before the source begins signaling back the feedback it gets about the loss. The packets of this first exchange are marked with FNE code point. The resetting of  feedback loop takes place essentially during the creation of the transport layer end-to-end connection.  Hence packets marked with FNE corresponds to SYN and SYN-ACK messages in TCP terminology. But, the restart of the loop could also take place after a long idle period or new attempts for retransmission after successive timeouts.

Many advantages could be drawn from this signaling. First, it eliminates the network need for inspecting higher layer fields to identify connection establishment. But most of all, a new  association between the end hosts, and visible at the network layer, could be established using the FNE-marked packets. This association could be used by the middleboxes to allocate state for different functions like control admission and traffic shaping. This association similar to the network flowlet used in FLARE (see 2.1.4) for traffic splitting presents the same advantages since it allows to divide transport flows in a succession of packet streams of smaller granularity. However, it presents an additional advantage since both network and transport layer are aware of it and it could be used for the latter to explicitly express that it desires packets belonging to this stream to be routed all on the same path.

Code Point
Explanation/Meaning
Not-LECT
Not Loss Exposure Capable transport
LECT
Loss Exposure Capable transport
LEx
Loss Experienced
FNE
Feedback Not Established

Echoing the Loss

Now that the feedback has being established, the end host could start revealing to the network the loss experienced in the last period of time. LEX is a simple form of Congestion Exposure that requires the end host to only mark retransmitted packets with the (LEx) code point. For the rest of the non retransmitted packets, the (LECT) code point is used. Coupled with the ECN marking, the network is in possession of two signals of information: an end-to-end vision of the experienced losses by the end host over the path during the previous RTT thanks to LEX, and the  current congestion experience at the upstream of each node that ECN markings provide. Hence the network could estimate rest-of-path congestion. But mostly the network will be able to accurately estimate the end-to-end loss experienced in every path by simply dividing the number of bytes marked with (LEx) code point,  by the total number of bytes marked either LECT or LEx. As we are going to see in the balancing section, the information of the loss experienced in the different paths will be used to decide the amount of flows that will in every path. Another potential advantage that could be drawn from this marking is that the network could prioritize retransmitted packets.

Analysis of the LEX protocol and comparison with re-ECN

LEX reuses many of the concepts that the congestion exposure protocol re-ECN has brought. The use of FNE packets by the network for allocating the state is one example of the similarities. Also by coupling LEX with ECN markings the network will be able to estimate the rest-of-path congestion and hence a metric will be available for both charging the customers on the congestion that they are causing and evaluating the service offered by the network. However, the drawback of this scheme is that it requires the aggregation of the packet streams to be carried out close to the source while the policing should be placed close to the receiver. This is not a significant problem, since the focus of the PREFLEX is to allow stub domains to balance the congestion. Moreover, LEX and re-ECN could coexist together: LEX code points could be easily added to the re-ECN mechanism since three of them have equivalent code points in re-ECN while LEx could be attributed to the currently unused code point in re-ECN. Hence LEX could be a complement to re-ECN. Thus, LEX could be considered a bridge for re-ECN until congestion notification is widely deployed in the network.

\section{Path Re-Feedback}

The second component of the architecture is PREF mechanism that allows stub domains to inform the end host of its path preference, From one hand it allows end hosts to be aware of the path the packets are taking. It also provides another solution for the problem of traffic splitting (see 2.1.3), since end hosts knowing the network path preference could take in charge the responsibility of assigning packets to the path. Using FNE packets to divide the flows to smaller granularity permits to achieve an accurate splitting of the traffic while preventing end hosts to suffer from packet re-ordering. PREF uses incoming FNE packets to trigger the path selection process for outgoing paths. This means that receiver will be left with the freedom of deciding when it is best to request a new path. The efficiency of the scheme is dependent on the strategy adopted by the receiver for this matter. It also could be a source of vulnerability for the network since FNE packets requires from the network  additional resources, both computational -e.g selecting a path- or in memory -keeping a state-. Therefore, the network might need to provision mechanism to limit the FNE packets rate.

The implementation of the path re-feedback could take place in the DiffServ field of the IPv4 header and more precisely the bits reserved for local use. So, once the edge network re-feedback its path preference the end host should tag all the subsequent packets with the path value.
It is worthy to mention that eve if the two architecture components are functionally separate their cooperation is necessary to achieve the target of congestion balancing. In the next section we will see how the architecture manages to optimally balance the congestion over the available paths.

\section{Balancing by PREFLEX}

\subsection{Algorithm requirements}
The algorithm has as a key target to succeed in balancing the congestion equally over the available routes. To this optimization goal, the design of PREFLEX architecture implies some additional requirements: 

First, the path probing mechanism in PREFLEX relies on the loss experienced on the traffic that the end user sends over each path. Obviously, if  there is no traffic sent over the path there will be no feedback about the path congestion. Hence, the algorithm should attribute a minimum share of the traffic over all the paths,
The algorithm shouldn't make assumptions about the actual distribution of the flows based on the previous splits: since the path selection process is triggered by the receiver, the algorithm shouldn't expect that all the flows had have seen a new path assigned between two decision intervals. Only, the flows that had sent FNE packets will be distributed according to the last desired split, while the long living flows that didn't ask for a new path will keep their old path and thus their distribution is based on previous splits. The actual flowlets distribution is unknown and a combination of the two.

\subsection{Algorithm inputs and outputs}

PREFLEX balancing algorithm requires to keep some information about each of the available paths. These information are contained in the routing entries that are organized in tables for each egress destination $d$. The nature of this destination depends on the context were PREFLEX is invoked so it may correspond to a set of IP destinations addresses or prefixes, or it may correspond to effectively the  address of the egress gateway to reach an external domain in the case of intra domain Traffic Engineering. 

A set of available routes $P_{d}$ is associated with destination $d$. For each route $i$ from the previous set, a table count of LEX markings is kept. In particular, $L_{di}$ be the number of bytes marked with the loss experienced codepoint and sent through route $i$ in the previous time period and $T_{di}$ denotes the total number of accounted bytes on routes $i$ for the same period of time. 

The two inputs will be used by the algorithm to determine the split of destination $d$ traffic over its permitted routes. The split indicates $f_{di}$, the fraction of the flowlets to destination $d$ that should be assigned to path $i$. 

\subsection{Algorithm description}

The flowlet split for each destination could be calculated according to three approaches. An equalization mode where the tendency is to equally distribute the traffic over the flows, a conservative mode where the importance is given to the stability of the algorithm and to prevent overreacting for small fluctuations. And the last one is to balances the losses obviously.
Hence, the final split is obtained as the combination of the splits calculated according to the three tendencies:

\begin{equation}
f'_{i} = \beta_{E}f'(E)_{i} + \beta_{C}f'(C)_{i} + \beta_{L}f'(L)_{i}
\end{equation}

Using the dash notation for the same quantities in the next time period, where $f'(E)_{i}$, $f'(C)i_{i}$ and $f'(L)_{i}$ consecutively denotes the split calculated according to “equalization” approach $E$, “conservative” approach $C$, and “loss driven” approach $L$. Similarly, $\beta_{E}$, $\beta_{C}$ and $\beta_{L}$ denotes positive factors associated with each mode and that verifies $\beta_{E}+\beta_{C}+\beta_{L} = 1$. 

Now by definition $f'(E)_{i} = 1/N$. For the other two splits we need an additional assumption. As explained in 3.3.2,  the balancer doesn't necessary achieve the desired split by the next period of time. Hence the actual flow distribution will be estimated as $T_{di}/ T_{d}$, the fraction of bytes to destination $d$ that takes route $i$. $T_{d} = \sum_{i \in P_{d}}T_{di}$ denotes the total bytes for destination $d$,  and in a similar way let $L_{d} = \sum_{i \in P_{d}}L_{di}$ the total losses bytes for destination d. The accuracy of this assumption could be explained by the fact that flows experiencing the same loss rate will have equal transmission rates (Mathis formula).

Hence, for the “conservative” approach $f'(C)_{i} = T_{i}/T$. While for the “loss driven” approach we will choose the split as  $f'(L)_{i} = T'_{i}/T''$ so that the loss rate on the different routes is equal.

The loss ratio is defined as $\rho_{di} = L_{di}/T_{di}$ for all the permitted routes $i.$ The loss rate is an unknown function of $T_{i}$ and the link bandwidth $B_{i}$. However, it is reasonable to assume that the loss rate is increasing with $T_{i}$ and decreasing with $B_{i}$. Hence, what ever the true function is, we could assume that in a small region around the current values of $T_{i}$ and $B_{i}$, it is locally linear $\rho_{i} = k_{i}T_{i}/B_{i}$, where $k_{i}$ is an unknown constant. Substituting gives
\begin{equation}
B_{i}/k_{i} = T_{i}^{2}/L_{i}
\end{equation}

So for the next period of time, loss rate are $\rho'_{i} = T'_{i}/L'_{i} = k'_{i}T'{i}/B'{i}$. Hence, to make the loss rate equal for all routes, we should choose $T_{i}$ that verifies $L'_{i}/T'_{i}=C$, where C is an unknown constant. Therefore we should take  $T'_{i}= CB'_{i}/k'_{i}$ and by assuming we are still near the locally linear region we get  $T'_{i}= CB_{i}/k_{i}$ and hence using (1) we get  $T'_{i}= CT_{i}^{2}/L_{i}$. Summing over i will give us: $T' = C \sum_{i} T_{i}^2/L_{i}$. Assuming that the traffic on average stayed the same we get the value of C and and though: 

\begin{equation}
T'_{i} = \frac {TT_{i}^{2}} {L_{i} \sum_{i} (T_{i}^2/L_{i})}
\end{equation}

Hence, the "loss driven" split, that needs to be set to achieve the $T'_{i}$ number of bytes on path i, is:

\begin{equation}
f'(L)_{i} = \frac{T'_{i}}{T} =  \frac {T_{i}^{2}} {L_{i} \sum_{i} (T_{i}^2/L_{i})}
\end{equation}

